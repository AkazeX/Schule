
 \documentclass[11pt, twocolumn, a4paper]{scrartcl}


\usepackage[ngerman]{babel}	% Worttrennung
\usepackage[utf8]{inputenc}	% Wird für die direkte Eingabe von Umlauten gebraucht
\usepackage[T1]{fontenc}		% Trennung von Umlauten
\usepackage{lmodern}		% Latin Modern Schrift
\usepackage{caption}			% Überschrift in Fliessumgebung
\usepackage{graphicx}		% Das Standardpaket zum Einbinden von Bildern / Grafiken
\usepackage{amsmath}		% Zusätzliche mathematische Umgebungen
\usepackage{amssymb}		% Zusätzliche mathematische Symbole
\usepackage{amsfonts}		% Symbole, Schriften
\usepackage{siunitx}			% Darstellung von SI Einheiten
\usepackage{hyperref}		% Erstellt Verweise innerhalb und nach außerhalb eines PDF Dokumentes
\usepackage{pdfpages}		% Einbinden von PDF Dateien zum Beispiel weitere bereits fertige PDFs in ein neues PDF einfügen
\usepackage{longtable}		% Für Tabellen die länger als eine Seite Lang sind
\usepackage{fancybox}		% Box für Formeln
\usepackage{fancyhdr}		% Seiten schöner gestalten, insbesondere Kopf- und Fußzeile
\usepackage{booktabs}		% Unterschiedlich dicke Linien in Tabellen
\hypersetup{pdfborder={0 0 0}}	% Keine Kästchen im Inhaltsverzeichnis




%Titel%%%%%%%%%%%%%%%%%%%%%%%%%%%%%%%%%%%
\title{Wirtschaft Angewandt\\
\begin{large}Leistungsnachweis Wirtschaft - Mathematik interdisziplin{\"a}r\end{large}}
\date{23.05.2014}
\author{Andreas Schneider \thanks{Berufsbildung Baden}}



%% Dokument Beginn %%%%%%%%%%%%%%%%%%%%%%%%%%%%%%%%%%%%%%%%%%%%%%%%%%%%%%%%
\begin{document}

\maketitle
\begin{small}
\setcounter{tocdepth}{1}
\tableofcontents 
\end{small}


%Text nach dem Prozentzeichen erscheint nicht im fertigen Dokument
%Sie können das Prozentzeichen verwenden um zu kommentieren und zu unterteilen

%HIER KOMMT IHR TEXT HIN

%Kapitel%%%%%%%%%%%%%%%%%%%%%%%%%%%%%%%%%%%%%%%%%%%%%%
\section{Kapitel} %Kapitel
\label{sec:Einleitung}%Damit können Sie auf das Kapitel bezug nehmen. \ref{sec:Einleitung} auf Seite \pageref{sec:Einleitung}



\subsection{Unterkapitel}%Unterkapitel

%Bilder einfügen
%\begin{figure}
%\includegraphics[width=0.5\textwidth]{Bilanz.jpg}
%\caption{Quelle: \protect\url{http://www.rwi.uzh.ch/elt-lst-vogt/gesellschaftsrecht1/revision_rechn/de/html/bilanz_gliederung.html}}
%\end{figure}

%Tabelle
\begin{tabular}{p{3cm}p{3cm}}
\toprule
\textbf{Fachbegriff} & \textbf{Beschreibung} \\
\midrule
Debitoren&Guthaben gegenüber Kunden aus Verkäufen die nicht sofort bar bezahlt werden.\\
\midrule
Kreditoren&Schulden bei Lieferanten aus Verkäufen die nicht sofort bar bezahlt werden.\\
\midrule
Umlaufvermögen& Flüssige Mittel und Vermögensanteile die innerhalb eines Jahres zu Geld gemacht werden können.\\
\midrule
Anlagevermögen&Teile des Vermögens einer Unternehmung, die nicht zur Veräusserung bestimmt sind.\\
\bottomrule
\end{tabular}





%von zweizeiligem Text auf einzeiligen wechseln
\onecolumn
\section{Angewandte Aufgaben}
\subsection{Macht-Index}



%Aufzählen
\begin{itemize}
\item Berechnen Sie das arithmetische Mittel der Benzinpreise.
\item Überprüfen Sie diesen Wert durch Nachrechnen, indem Sie Berechnen, wie viele Liter getankt wurden.
\item Berechnen Sie das harmonische Mittel.
\item Welcher Mittelwert eignet sich besser?
\end{itemize}
%Für die Mathematik brauchen Sie das Dollar-Zeichen am Anfang und am Ende
$a^2 \cdot b^2 = c^2 \Rightarrow c=\sqrt{a^2 \cdot b^2}$
%Oder
\begin{equation}
\label{eq:1.8.7}
\frac{Zähler}{Nenner}
\end{equation}


%% Dokument ENDE %%%%%%%%%%%%%%%%%%%%%%%%%%%%%%%%%%%%%%%%%%%%%%%%%%%%%%%%%%
\end{document}

